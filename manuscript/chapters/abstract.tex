\chapter*{Abstract}
\addcontentsline{toc}{chapter}{Abstract}
This thesis presents a novel application of Statistics and Machine Learning techniques to analyze the Qur'\=an, complementing traditional methodologies while approaching the sacred text with appropriate respect and care. The research examines structural and linguistic patterns through multiple analytical lenses. Descriptive statistics reveal distinct clusters between \arb[trans]{makkiyyaT} \arb{makkiyyaT} and \arb[trans]{madaniyyaT} \arb{madaniyyaT} \arb[trans]{suwar} \arb{suwar}, with the latter demonstrating higher median values and greater variability. Morphological analysis of the root word \arb[trans]{Alh} \arb[novoc]{Alh} shows its common form \arb[trans]{'l-lahi} \arb[fullvoc]{'l-lahi} predominantly appears in \arb[trans]{madaniyyaT} \arb{madaniyyaT} chapters, while rarer forms (\arb[trans]{--|"'Alihati} \arb[fullvoc]{--|"'Alihati} and \arb[trans]{'alihatu} \arb[fullvoc]{'alihatu}) are exclusive to \arb[trans]{makkiyyaT} \arb{makkiyyaT} texts. Examination of manuscripts from 660-1000 CE from the Corpus Coranicum project confirms the preservation of these morphological patterns, supporting the integrity of oral transmission. For rhythmic analysis, the paper propose visualization techniques including line plots, heatmaps, rhythmic graphs, and histogram density plots, revealing strong patterns where approximately 70\% of \arb[trans]{'ayaT} \arb{'ayaT} endings feature transitions from short to long vowels. The paper further investigate concentric structures in \arb[trans]{sUraTu 'l-baqara} \arb{sUraTu 'l-baqara} using a novel Genetic Algorithm approach to objectively identify structural borders, employing CL-AraBERT for word embeddings with Cosine Distance metrics. The resulting optimal structural borders ($A$, $B$, $C$, $D$, $C^*$, $B^*$, $A^*$) were thematically analyzed using GPT-4o, confirming the concentric arrangement. This research demonstrates the value of computational methods, particularly the Julia programming language with libraries like QuranTree.jl and Yunir.jl, in revealing new insights into this complex, multi-layered text while respecting its sanctity and significance.
