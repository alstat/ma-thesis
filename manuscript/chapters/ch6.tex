
\chapter{Methodology}
This chapter is organized as follows: Section \ref{sec:topic_modeling_method} will discuss the concept of Topic Modeling, Section \ref{sec:llm_method} will discuss the concept of Large Language Models, and finally Section \ref{sec:code_setup} will discuss how to implement this in Julia programming language.
\section{Structural Analysis}
\subsection{Bayesian Statistical Models}

\section{Rhythmic Analysis and Modeling}
Analyzing the rhyme starts with collection of the ending syllable of the last word of each \arb[trans]{'ayaT} \arb{'ayaT} in the \arb[trans]{sUraT} \arb{sUraT}. This can be done per \arb[trans]{sUraT} \arb{sUraT} or across the Qur'\=an. For this study, both are done to see if there are interesting patterns. 

The analysis is done by visualizing the signature of the rhythms, and also modeling the transition probabilities between the ending syllables. That is, answering the question like what is the probability of getting a ending rhyme of \arb[trans]{iyn} \arb{iyn} given the previous verse ends with \arb[trans]{uwn} \arb{uwn}.

Therefore, each \arb[trans]{sUraT} \arb{sUraT} will have its own probabilistic graphical model, which can be analyzed by clustering similar rhymings of \arb[trans]{sUraT} \arb{sUraT}.

The signature can be plotted using line chart, and the verses can be divided into juz. So that, there will be 30 rows for 30 juz.
\subsection{Bayesian Network Model}
\section{Verse Division Analysis and Modeling}
In theory, verse division depends on the rhyme and the length of the verse. For this to be modeled, a discrete-time markov chain can be used with transition probability modeled by a Geometric distribution. The idea is that an \arb[trans]{'ayaT} \arb{'ayaT} will end if enough number of words have been recited and that the rhyme is achieved. The question is how many number of words to be recited for it to be enough to transition to the next verse. The states here can be defined as the ending rhyming syllable used in the previous section. 
\subsection{Bayesian Discrete-Time Markov Model}
\section{Thematic Analysis}
This section 
\subsection{Topic Modeling}\label{sec:topic_modeling_method}
As presented in Chapter \ref{ch:introduction}, the first objective is to extract the thematic themes of \textit{s\=urahs} \arb{sUr} with at least 1000 words. In Statistics and Machine Learning, this task is called Topic Modeling. There are several ways to do this, but the popular methodology is to use the Latent Dirichlet Allocation (LDA) discussed in the next section.
\subsection{Latent Dirichlet Allocation}\label{sec:lda}
Latent Dirichlet Allocation (LDA) is a Statistical methodology that is based on Bayesian inference \cite{bayes,laplace1986}. It is a generative probabilisitic model for collection of discrete data such as text corpora \cite{blei2003latent}. The main formula is defined below:
\begin{defnx}[Latent Dirichlet Allocation]
Let $\mathbf{W},\mathbf{Z},\boldsymbol{\theta},\boldsymbol{\varphi}$ be the random variables, and let $\alpha$ and $\beta$ be the hyper-parameters, then the probability of generating a document is
\begin{equation}
    \mathbb{P}(\mathbf{W},\mathbf{Z},\boldsymbol{\theta},\boldsymbol{\varphi})=\prod_{j=1}^m\mathbb{P}(\boldsymbol{\theta}_j;\alpha)\prod_{i=1}^{k}\mathbb{P}(\boldsymbol{\varphi};\beta)\prod_{t=1}^{n}\mathbb{P}(\mathbf{Z}_{j,t}|\boldsymbol{\theta}_j)\mathbb{P}(\mathbf{W}_{j,t}|\boldsymbol{\varphi}_{\mathbf{Z}_{j,t}})
\end{equation}
\end{defnx}
\subsection{Large Language Models}\label{sec:llm_method}
Generative Artificial Intelligence or GenAI for short has been making waves on its effectiveness to generate texts, images, audio, video, etc. It has elevated humanity to a new level of capability. However, behind this amazing capabilities is that GenAI is by design a mathematical formula that are called \textit{model}. There are several types of \textit{models}, and one of those is the Large Language Model (LLM). The following section will discuss what LLM is and its mathematical formulation.
\subsection{Bidirectional Encoder Representation from Transformers}
BERT or Bidirectional Encoder Representation from Transformers model is a large language model proposed by \citeA{devlin2018bert}. From the name itself, it is based on the Transformer model architecture (\textit{see} discussion in Section \ref{sec:transformers}) in that it only uses the Encoder layer, and stack it together. BERT was pre-trained on large corpus of text using two unsupervised (\textit{see} Section \ref{sec:unsupervised_models}) tasks, and these are:
\begin{enumerate}
    \item \textit{Masked Language Modeling (MLM)} - tokens (\textit{see} Section \ref{sec:text_tokenization}) are randomly masked in the input and trains the model to predict these masked tokens based on the surrounding context.
    \item \textit{Next Sentence Prediction (NSP)} - tains the model to understand the relationship between two sentences by predicting if one sentence follows the other.
\end{enumerate}
After pre-training the model, BERT can then be fine-tuned on specific tasks like question answering, sentiment analysis, and more with relatively smaller datasets. With that, BERT works as follows: 
\begin{enumerate}
    \item \textit{Input Representation} - BERT takes tokenized text as input, which includes a pair of sentences. The input is converted into tokens, added with special tokens like [CLS] (classification token at the beginning) and [SEP] (separator token between sentences).
    \item \textit{Embedding Layer} - The tokens are converted into embeddings which are the sum of token embeddings, segment embeddings, and position embeddings.
    \item \textit{Encoder Layers} - The embeddings are then passed through multiple layers of bidirectional Transformer encoders (\textit{see} Section \ref{sec:transformers}), which apply self-attention mechanisms to generate contextualized representations for each token.
    \item \textit{Output} - The final hidden states from the encoder layers are used for different tasks:
    \begin{itemize}
        \item The [CLS] token’s representation can be used for classification tasks.
        \item The representations of other tokens can be used for tasks like named entity recognition (NER) or question answering.
    \end{itemize}
\end{enumerate}
There are several applications of BERT model, but for this paper it will be used for Topic Modeling and Text Summarization of the Qur'\=an. In particular, CL-AraBERT model by will be used for extracting embeddings of the Qur'\=anic words for further analysis.
\subsection{Generative Pre-Trained Transformer}
GPT or Generative Pre-Trained Transformer is another large language model proposed by \cite{radford2018improving}. From the name itself and like BERT, GPT is based on the Transformer model \cite{vaswani2017attention}, \textit{see} Section \ref{sec:transformers}. Unlike BERT though, GPT uses the decoder layer of the Transformer model and stacks it multiple times. This is the model that is powering the ChatGPT\footnote{\url{https://chat.openai.com/}} of OpenAI and also Claude AI\footnote{\url{https://claude.ai/}} of Anthropic\footnote{\url{https://anthropic.com/}}.

GPT models like those powering ChatGPT were pre-trained on large corpora by going through the sequence of the texts in \textit{unidirection}, which is contrary to the \textit{bidirectional} approach of BERT model. As such, the GPT models excel in generating text and performing tasks that require producing coherent sequences of words, in applications like text completion and creative writing. Whereas BERT is technically effective for tasks requiring deep contextual understanding such as text classification and named-entity recognition.

For this paper, the 
\section{Concentrism Formulation}
\subsection{Cosine Similarity}
\subsection{Kullback-Leibler Divergence}
\section{Integrating Other Islamic Literatures}
\subsection{Retrieval-Augmented Generation}
The problem with LLM is that it was only trained on huge but limited data, and is therefore not able to infer what should be the context when asked.
\section{Programming Languages Setup}\label{sec:code_setup}
This section will discuss the coding setup. As mentioned in Chapter \ref{ch:introduction}, the main programming language to use is Julia. As such, it is necessary to present where the codes will be stored so that readers are able to reproduce it. All of the codes will be saved in Github repository accessible through the following link:
\begin{center}
    \url{https://github.com/alstat/ma-thesis/tree/main/codes}
\end{center}
\subsection{Julia}
\subsection{Python}
\subsection{R}