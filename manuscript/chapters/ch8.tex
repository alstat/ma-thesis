\chapter{Future Research}\label{ch:future_research}
The paper, though with short but ambitious title, gives the impression of covering all aspects of text analytics of the Qur'\=an. While this is possible, but the limited research in this approach necessitate the author to cover first the fundamendals of text analytics, and to show the possibilities of using Statistics and Machine Learning for analyzing the Qur'\=an. This new perspective and approach have opened up new avenues for further research, and indeed there are many areas that can be explored in the future. Some of these areas are discussed below.

\section{Morphological Analysis}
The morphological analysis of the Qur'\=an is another area that can be explored in more depth. The paper has shown that the morphological features of the Qur'\=an can be used to extract the root words and their morphological forms, but there are many other aspects of morphological analysis that can be studied. For example, the relationship between the morphological features and the meaning of the text can be explored in more depth. This could involve analyzing how changes in morphology correspond to changes in meaning or theme within the text. Additionally, the morphological features can be used to analyze the different styles of writing in the Qur'\=an, and how these styles correspond to the themes and topics within the text.

\section{Rhythmic Analysis}
The rhythmic analysis of the Qur'\=an is a complex and multi-layered topic that can be explored in many different ways. The paper has proposed a methodology of different ways to visualize the rhythmic patterns including a basic computation of transition probabilities, but there are many other aspects of rhythmic analysis that can be studied. For example, the relationship between the rhythmic patterns and the meaning of the text can be explored in more depth. This could involve analyzing how changes in rhythm correspond to changes in meaning or theme within the text. Additionally, the rhythmic encoding only captures three types of syllables, those with short vowel, those with long vowel, and those with \arb[trans]{maddaT} \arb{maddaT}. These three only captures the bare minimum of the richness of the Arabic language. The different sounds of the Arabic letters, such as the emphatic letters, the guttural letters, and the labial letters, can also be included in the rhythmic encoding. This would allow for a more comprehensive analysis of the rhythmic patterns in the Qur'\=an. The cadence associated by these letters and its diacritics can also be encoded into the rhythmic patterns. 

Another approach is that, instead of looking at the texts itself for extracting the rhythmic patterns, one can look at the audio recitation of the Qur'\=an by Muslim reciters. This approach would involve analyzing the audio recordings of the Qur'\=an recitation, and extracting the rhythmic patterns from the audio data. This could involve using techniques such as Fourier analysis or wavelet analysis to analyze the frequency and amplitude of the audio signals. The rhythmic patterns extracted from the audio data could then be compared to the rhythmic patterns extracted from the text data to form a more comprehensive understanding of the rhythmic patterns in the Qur'\=an. This approach would also allow for the analysis of the different styles of recitation, and how these styles correspond to the rhythmic patterns in the text.

\section{Other Symmetric Structures}
The paper only explored the concentric structures in the Qur'\=an using the Genetic Algorithm, but the other patterns like \textit{parallelism} and \textit{chiasmus} can also be explored. 

\section{Other Research Areas}
The methodology used in this paper can be extended to also studying other Islamic texts, such as the \arb[trans]{'a.hAdI_t} \arb{'a.hAdI_t}, and other Islamic texts which are also available through the Kitab.jl \cite{al_ahmadgaid_b_asaad_kitab} Julia library, which can form networks of corpus that can be analyzed using the same techniques.