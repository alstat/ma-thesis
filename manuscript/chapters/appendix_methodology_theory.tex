\chapter{Mathematics of the Methodology}\label{app:math_methodology}
This appendix provides the mathematics behind the methodology in Chapter \ref{ch:methodology}. 

\section{Basic Statistics}
The following definitions are the mathematics behind the basic statistics used in this study.

\begin{defn}[Mean]\label{defn:mean_method}
    Let $x\in\mathbb{R}, n\in\mathbb{N}$ s.t. for any $D:=\{x_1,\cdots,x_n\}$, the \textit{mean}, notated as $\bar{x}$ is given by:
    \begin{equation}
        \bar{x}:=\frac{1}{n}\sum_{\forall i}x_i,\quad i\in\mathbb{N}.
    \end{equation}
\end{defn}
\begin{defn}[Median]
    From Definition \ref{defn:mean_method}, the \textit{median}, notated as $\tilde{x}$, is given by:
    \begin{equation}
        \tilde{x}:=\begin{cases}
            x_{((n-1)/2)+1)},& n\operatorname{mod}2 \neq 0,\\[0.3cm]
            \displaystyle\frac{x_{(n/2)}+x_{(n/2+1)}}{2},&\text{otherwise}
        \end{cases},
    \end{equation}  
    where $x_{(i)}$ is the $i$th \textit{order statistics} of $x$.
\end{defn}
\begin{defn}[Variance]\label{defn:method_variance}
    From Definition \ref{defn:mean_method}, the \textit{variance}, notated as $\sigma^2$ is given by:
    \begin{equation}
        \sigma^2:=\frac{1}{n}\sum_{\forall i}(x_i-\bar{x}_i)^2
    \end{equation}
\end{defn}
\begin{defn}[Standard Deviation]
    From Definition \ref{defn:method_variance}, the \textit{standard deviation}, notated as $\sigma$ is given by $\sigma:=\sqrt{\sigma^2}$.
\end{defn}
\begin{defn}[$q$th Sample Quantile]
    From Definition \ref{defn:mean_method}, the $q$th \textit{sample quantile}, notated as $Q(p)$ where $p$ is the proportion, is given by
    \begin{equation}
        Q(p;\alpha,\beta):=(1-\gamma)x_{(i;\alpha,\beta)} + \gamma x_{(i+1;\alpha,\beta)},
    \end{equation}
    where $x_{(i)}$ is the $i$th \textit{order statistics} of $x$, $i:=\lfloor np+m\rfloor$, $m:=\alpha+p(1-\alpha-\beta)$, and $\gamma:=np+m-j$.
\end{defn}
\begin{remark}
    The $q$th sample quantile is the formula used in Julia programming language, and by default $\alpha=1$ and $\beta=1$.
\end{remark}

\section{Neural Networks and Large Language Models}
This paper uses CL-AraBERT model \cite{MALHAS2022103068} for extracting the word embeddings of the Qur'\=anic words, and also uses Generative Pre-trained Transformer (GPT) models \cite{radford2018improving} for extracting Thematic Analysis. The methodology as to how this works is discussed in Appendix \ref{app:neural_networks}.