\chapter{Theoretical and Conceptual Frameworks}\label{ch:tf-cf}
The theories from the literatures reviewed in the preceding chapter will be discussed in this chapter. In addition to this, the solutions will be presented at a high level through the conceptual framework, where the details will be discussed in the methodologies.
\section{Theoretical Framework}
As mentioned from Chapter \ref{ch:rrl}, when it comes to statistical analyses of the morphological characteristics of the Qur'\=an corpus \shortcite{dukes-habash-2010-morphological}, most of it were simply comparison to the other corpora, none has yet looked into it in terms of statistical analyses of the positional statistics of these morphological features like the parts-of-speech, and name of entities like God's names and prophets names. The interest in studying this statistics will also help in understanding the structure of the Qur'\=an as it should give us an idea on how these topics or names of these entities are positioned in the Qur'\=an.

Moreover, in Chapter \ref{ch:rrl}, the work of \shortciteA{siddiqui2013} used Latent Dirichlet Allocation (LDA) on 24 \arb[trans]{sUwar} \arb{sUwar} using Term Frequency-Inverse Document Frequency (TF-IDF) as the word embeddings. While this procedure does work, but the recent development on Large Language Models (LLMs) has resulted into improved word embeddings, such as the Bidirectional Encoding Representation from Transformers (BERT) models \shortcite{devlin2018bert}. BERT models, discussed in Section \ref{sec:bert}, are superior than TF-IDF since, by design, it has lower dimensional embedding and does not depend on the number of vocabularies. Semantically, it is also better as it uses a \textit{attention mechanism}\footnote{\textit{See} Section \ref{sec:attention-mechanism}}, which takes into account the surrounding contexts of the given embedding. Further, while LDA is an effective statistical methodology for topic modeling, this paper will also look into BERTopic by \shortciteA{grootendorst2022bertopic}, which is a topic modeling model based on BERT and was found to provide better results compared to other algorithms according to \shortciteA{egger2022topic}.

Further, when it comes to \textit{rhythmic analyses} of the Qur'\=an's texts, none so far has looked into it from the perspective of statistical analyses. As such, this paper will also provide a mathematical formulation of the said analyses, and also its applications to statistical methodologies. This is the first to do so, and methods for this analyses will include cluster analyses to understand if there are groupings of \arb[trans]{sUwar} \arb{sUwar} with respect to this statistics. Other methodologies will revolve around descriptive statistics to describe the rhythmic signatures of \arb[trans]{sUwar} \arb{sUwar} with respect to its place of revelation, that is, either \arb[trans]{makkiyyaT} \arb{makkiyyaT} or \arb[trans]{madaniyyaT} \arb{madaniyyaT}.

Moving on, with regards to the \textit{theory of concentrism}, the same situation as the \textit{rhythmic analyses} of the Qur'\=an, in that the paper will be the first to provide a mathematical formulation of the theory. In addition to this, one of the critics on the works of \shortciteA{farrin2014structure}, is that the choice of the structural border between groups of concentric pattern are often not straightforward or no clear ruling, \textit{see} \shortciteA{sinai2017review}. As such, it this paper will propose an optimization algorithm that will find the optimal structural borders of the groups of concentric pattern. This optimization algorithm will be the Bayesian optimization discussed in Section \ref{sec:bayes-opt}.

Finally, when it comes to Retrieval-Augmented Generation (RAG), the work of \shortcite{alan2024rag} uses GPT3.5 Turbo as the LLM model augmented with Islamic texts. For this study, the use of RAG is not only limited to.


\section{Conceptual Framework}