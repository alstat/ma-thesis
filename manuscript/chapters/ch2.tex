\chapter{REVIEW OF RELATED LITERATURE}\label{ch:rrl}

As mentioned in the previous chapter, the earliest paper on Qur'\=anic studies using computers was likely the work of Rashad Khalifa in 1968\footnote{\url{https://www.masjidtucson.org/quran/miracle/a_profound_miracle_sura68nun133.html}}, which led to one of his books entitled `The Computer Speaks: God's Message to the World' (\textit{see} \citeA{rashad1981}). While the work of Rashad started by studying the mystery letters at the beginning of some \textit{s\=urahs} \arb{sUr} (for example Qur'\=an 2:1, 3:1, 7:1, etc.), it quickly went on to cover what he called other \textit{mathematical miracles}, all of which are covered in \citeA{rashad1981}. His work was mainly based on counting the positional statistics of selected entities like names of God relative to other names or words in the Qur'\=an, and by showing through counting and simple mathematics of factoring some totals, he demonstrated interesting matches among these statistics. He was able to do this by transliterating the Qur'\=an into alpha-numeric typesets for easy parsing by the computers of that time. His findings led him to generalize the claim that God revealed His words through these "mathematical" patterns throughout the Qur'\=an, and that those verses that were inconsistent and did not conform to these discovered "mathematical" patterns led him to extensive investigation of the said verses, and he concluded that those could be or surely were insertions that should not have been in the Qur'\=an in the first place. There are two verses that were inconsistent according to Rashad, and he called these verses \textit{false verses}\footnote{\url{https://submission.org/App24.html}}; these are the last two \arb[trans]{'AyAt} \arb{'AyAt} of \arb[trans]{sUraTu 'l-tAwbaT} \arb{sUraTu 'l-tAwbaT} or The Chapter of \textit{Repentance}. These two verses were removed in Rashad's Qur'\=an translation\footnote{\url{https://www.masjidtucson.org/quran/frames/}}. Rashad believed so strongly in his findings that he proclaimed himself to be a messenger\footnote{\url{https://www.masjidtucson.org/submission/faq/rashad_khalifa_summary.html}} with these new findings and that the Qur'\=an of today should conform to his discovered "mathematical" patterns. This self-proclamation led to his assassination.

What the above tells us is that early on, with the advent of computers, there was already interest in trying to understand the structural characteristics of the Qur'\=an. This is because the Qur'\=an is not like the Bible, which is arranged chronologically, but rather is arranged in generally decreasing numbers of \arb[trans]{'AyAt} \arb{'AyAt} per \arb[trans]{sUraT} \arb{sUraT} as shown in Figure \ref{fig:ayah_word_count}. This unique structure led to different literatures on understanding the Qur'\=an's design in terms of its unique presentation of topics, not only from Islamic studies researchers but also from computer scientists who were involved in Arabic Natural Language Processing studies. 

Among the pioneers in computational applications for the Qur'\=an is the work of \shortciteA{thabet2004}, who built a stemmer system for the Qur'\=an. A stemmer system is a system for trimming inflected words into their basic form, which grammatically means their root form. For example, in the English language, the root word for \textit{computational}, \textit{computer}, \textit{computation}, and \textit{computerize} is \textit{compute}. Therefore, the root of a word forms different \textit{stems} representing the different words. Hence, the idea of stemming is to trim these words into their basic form, so that it would be easy to do word clustering or grouping through word similarity. According to \citeA{thabet2004}, the rich morphology of the Qur'\=anic language or Classical Arabic makes it even more difficult to perform word stemming. Moving on, \citeA{thabet2005} builds on top of this stemming system and used it for tokenization of the Qur'\=anic words and for building a statistical methodology for clustering or grouping the chapters of the Qur'\=an. In particular, \citeA{thabet2005} used Agglomerative Hierarchical Clustering based on the Euclidean distance of the adjusted word frequency of a \arb[trans]{sUraT} \arb{sUraT}.

The work of \citeA{thabet2004} and \citeA{thabet2005} used a Qur'\=an corpus that was transliterated to Roman letters and symbols. Indeed, the growing interest in studying the Qur'\=an from the lens of Data Analysis and Natural Language Processing resulted in creating digital corpora of the Qur'\=an that capture the different aspects of its linguistic styles. Hence, a series of works by Sharaf and Atwell led to the following publications: \shortciteA{sharaf2009} studied knowledge representation of the Qur'\=an's verb valences using FrameNet frames, the output of which is a lexical database as a corpus of the Qur'\=an's verbs. Furthermore, the work of \shortciteA{sharaf2012} came up with a corpus for the annotations of the Qur'\=anic pronouns, which the authors named QurAna. Building on this work, \shortciteA{sharaf2012b} came up with a corpus for studying Qur'\=anic relatedness based on the commentary of Ibn Kathir \arb{ibn ka_tir}, and the authors named this corpus QurSim. Apart from Sharaf and Atwell, Dukes and Habash were also working on this, but specifically on the morphological annotations. The final and verified\footnote{The authors used online community collaboration for checking or verification of the morphological annotations. This was necessary since the Qur'\=an is a highly regarded book by Muslims, and there should be no mistakes in annotations.} work of the authors is \shortciteA{dukes-habash-2010-morphological}, which also led to other publications \shortcite{dukes2010online, dukes2013supervised, dukes2010dependency} related to this.

With the establishment of a morphologically annotated corpus for the Qur'\=an, the hope was to have further analyses of the said scripture using statistical and machine learning methodologies. As such, the work of \shortciteA{dukes-habash-2011-one, dukes2015statistical} was among the first to do so, where they constructed a statistical parser through machine learning. This was then followed by \shortcite{siddiqui2013}, who used the said corpus for topic modeling using Latent Dirichlet Allocation. The said study started with 114 \arb[trans]{sUwar} \arb{sUwar}, but after processing, it was reduced to 24 \arb[trans]{sUwar} \arb{sUwar} after considering \arb[trans]{sUwar} \arb{sUwar} with 1000 or more words. This was mainly due to the very sparse document term matrix for the Term Frequency - Inverse Document Frequency\footnote{\textit{See} Section \ref{sec:tf-idf}} (TF-IDF) embedding when considering all 114 \arb[trans]{sUwar} \arb{sUwar}. Aside from this, the rest have used the corpus by \shortciteA{dukes-habash-2010-morphological} as part of a benchmark for morphological analysis or for new lexicographic databases, for example \shortciteA{sabtan2017morphological, jarrar-hammouda-2024-qabas-open}.

For the case of \textit{rhythmic analysis} of the Qur'\=an, most of the literature has looked into the recitation of this rhythmic feature of the Qur'\=an. For example, the work of \shortciteA{shekha2013effects, samhani2022rhythms, abdullah2011effect} and \shortciteA{naqvi2020effect} have all investigated the effect of listening to Qur'\=an recitation on temporal Electroencephalogram (EEG) signals and on human physiology through Electrocardiogram (ECG) signals. The work of \shortciteA{shekha2013effects} subjected 11 students to listen to soft music, hard music, and Qur'\=anic recitation and collected the \textit{alpha} wave signals from EEG. The authors used t-test and ANOVA to compare the results from the three input audio sources. Their findings suggest that Qur'\=an recitation, both during open eyes or closed eyes while listening to the Qur'\=an, shows significant magnitude in \textit{alpha} wave electrodes. The same study was also conducted by \shortciteA{samhani2022rhythms}, but this time with listening to Arab News, Qur'\=anic recitation of \arb[trans]{sUraTu 'l-fAti.haT} \arb{sUraTu 'l-fAti.haT}, and simply resting (that is, not listening to any audio). The authors used Analysis of Variance (ANOVA) and Hierarchical Clustering for the resulting Beta oscillation data. Their study found that the said Beta rhythm synchronization of listening to \arb[trans]{sUraTu 'l-fAti.haT} \arb{sUraTu 'l-fAti.haT} is associated with verbal fluency, academic performance, social interaction, inhibitory function, movement planning, self-motivation, self-management, and reactivation of sensory features of memory trace as a highly activated cluster, followed by working memory, language processing, and decision making as a medially activated cluster. Another similar study was also done by \shortciteA{abdullah2011effect} with the same pattern of output, showing a good effect on the EEG of the patients. On the other hand, the work of \shortciteA{naqvi2020effect} investigated the effect of the Qur'\=an on the ECG. The authors used different classification models for classifying signals from Qur'\=an listening. 

While the rhythmic feature is the most obvious feature of the Qur'\=an, the structural arrangement of the topics and verses is not as obvious compared to any other books. Hence, there have been several attempts to understand these arrangements even from pre-modern Muslim scholars. The Basran writer al-Jahiz (d. 255/868 or 869) produced a work entitled \textit{The Composition of the Qur'an}. Even a scholar named Abu Bakr al-Nisaburi from the 4th century AH/10th century CE would ask such questions as, "Why is this verse next to the other one?" and, "What is the wisdom in the placement of this chapter next to this other one?" \shortcite{farrin2014structure}. According to \shortciteA{farrin2014structure}, since the 1980s, numerous scholars in the field have followed their lead and have begun to show clearly that individual chapters, both short and long, are indeed characterized by a high degree of unity. With that said, among the structures that were found is the \textit{theory of concentrism}, which is accordingly the structural pattern of the Qur'\=an. This was the proposition of Michael Cuypers\footnote{All of his works are written in French. Please refer to \shortciteA{farrin2014structure} for the list of articles written by Cuypers.}. As far as the author's knowledge goes, the most comprehensive exposition of the theory of \textit{concentrism} in English is the work of \shortciteA{farrin2014structure}. However, not a single paper has been written for a mathematical formulation of this that can be used for statistical analyses. 

Further, the publishing of the scans of old Qur'\=anic manuscripts as part of the Corpus Coranicum\footnote{\url{https://corpuscoranicum.de/en}} project has opened opportunities for other research studies. The most recent one is the work of \citeA{sidky2020}, which heavily used the said scans of extant manuscripts for studying the regionality of the Qur'\=anic codices. Sidky used over 50 Qur'\=anic extant manuscripts from Corpus Coranicum. In addition to this, \citeA{Rashwani_2020} conducted a critical review of the said project and concluded that the derived data---encodings of the orthographies by the project teams and other related derived data---are not reliable as they contain documentation errors such as incorrect vowelization. Rashwani calls for more rigorous methodology and better integration of traditional Islamic scholarship with modern critical approaches. For this study, however, and as pointed out already in Chapter \ref{ch:introduction}, the derived data encoded by the team from Corpus Coranicum are not used, but only the scans of the extant manuscripts. 

Finally, recent advances in Large Language Models (LLMs) have significantly improved natural language processing and word embeddings for numerical text representation. For Arabic texts, AraBERT \cite{antoun-etal-2020-arabert} is a popular choice, while CL-AraBERT \cite{MALHAS2022103068} model specializes in classical Arabic. This paper uses CL-AraBERT for numerical representation of Qur'\=anic texts for advanced mathematical algorithms.