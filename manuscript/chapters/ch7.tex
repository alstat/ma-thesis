\chapter{Summary and Conclusions}\label{ch:conclusion}
The paper has demonstrated the use of Statistics and Machine Learning for analyzing the Qur'\=an. As emphasized in the introduction of this paper, the Qur'\=an is a book that has been studied for centuries, and it is important to approach it with the utmost respect and care. The methodology used in this paper is not intended to replace traditional methods of studying the Qur'\=an, but rather to complement them by providing a new perspective on the text. The results of the analysis have shown that there are patterns and structures in the Qur'\=an that can be revealed through Statistical and Machine Learning techniques. These findings can help deepen our understanding of the text and its message.

The descriptive statistics of the count of characters, words, and \arb[trans]{'AyAt} \arb{'AyAt} following the \arb[trans]{'asbAb 'alnuzUl} \arb{'asbAb 'alnuzUl}, has shown that the \arb[trans]{makkiyyaT} \arb{makkiyyaT} and \arb[trans]{madaniyyaT} \arb{madaniyyaT}\newline\arb[trans]{suwar} \arb{suwar} forms separate cluster, with \arb[trans]{madaniyyaT} \arb{madaniyyaT} having higher median and higher variability compared to \arb[trans]{makkiyyaT} \arb{makkiyyaT}. 

Further, the Morphological Analysis of the root word \arb[trans]{Alh} \arb{Alh} in the Qur'\=an has shown that, its common morphology \arb[trans]{'l-llah} \arb{'l-llah} mostly occurs in \arb[trans]{madaniyyaT} \arb{madaniyyaT}, while its rare form only occur in \arb[trans]{makkiyyaT} \arb{makkiyyaT}. The investigation onto the extant manuscript dating from 660 - 710 CE to the 1000 CE shows that these morphologies have the correct \arb[trans]{rasm} \arb{rasm} or consonantal skeleton already from the early extant manuscripts, and has not been altered to the present times. This shows that there were no indication of changing the recitation from say possible editorial of the texts, instead it indicates that the orally transmitted Qur'\=an has been well preserved since there was no evidence on the addition or deletion on the texts that could have suggested a change in oral transmission.

Moreover, for Rhythmic Analysis of the Qur'\=an, this paper has proposed techniques for visualizing the rhythmic patterns, which include the use of line plots, heatmaps, rhythmic graph, and histogram with density plot. The results showed strong rhythmic patterns in the Qur'\=an, with transition probabilities that are consistent with the traditional understanding of the text. In particular, about 70\% of the time the reader will find the transition from short vowel to long vowel at the last two recited syllables of an \arb[trans]{'ayaT} \arb{'ayaT}.

Next, the paper also studied concentric structures in the Qur'\=an, particularly for \arb[trans]{sUraTu 'l-baqara} \arb{sUraTu 'l-baqara}, and found that it indeed show concentrism, but instead of subjectively selecting the structural borders for the concentric classes, the paper proposed a methodology for objectively finding the structural borders using the Genetic Algorithm. The algorithm was tested for two types of initial population of 1,000 structural borders, one being simulated from Dirichlet distribution and the other being simulated from discrete uniform distribution. The results showed that with proper padding, the discrete uniform population of structural borders produces a reasonable class borders with borders that are close to what \citeA{farrin2014structure} found. The algorithm used CL-AraBERT model for extracting the word embeddings of the Qur'\=anic words with Cosine Distance as the distance metric.

Furthermore, the themes from the optimal structural borders of Genetic Algorithm, $A\;(1-44)$, $B\;(45-86)$, $C\;(87-162)$, $D\;(163-163)$, $C^*\;(164-195)$, $B^*\;(196-236)$, $A^*\;(237-286)$, were extracted using OpenAI's GPT-4o model. The results showed that the themes do indeed form a concentric structure as validated by the author.

Finally, the paper has shown that the Qur'\=an is a complex and multi-layered text that can be analyzed using a variety of methods. The results of the analysis have provided new insights into the structure and meaning of the text, and have opened up new avenues for further research. The paper has also highlighted the importance of using a combination of traditional and modern methods in studying the Qur'\=an, and has shown that these methods can be used to complement each other in order to gain a deeper understanding of the text. In particular, apart from the Statistical and Machine Learning methods, the use of a programming language such as Julia with libraries like QuranTree.jl \cite{asaad2021qurantree} for interfacing the Qur'\=an, and Yunir.jl \cite{al_ahmadgaid_b_asaad_yunir} has shown to be very useful in analyzing the said Holy Book. It is indeed the author's hope that this paper will inspire further research in this area, by the adopting these techniques and computational methods to studying Islamic texts, not just the Qur'\=an but also \arb[trans]{'a.hAdI_t} \arb{'a.hAdI_t} and other Islamic texts which are also available through the Kitab.jl \cite{al_ahmadgaid_b_asaad_kitab} Julia library. In addition, the Corpus Coranicum project of digitizing most of the extant manuscript of the Qur'\=an has been invaluable for the advancement of the field.

Lastly, the author would like to emphasize again that just like other religious texts, the Qur'\=an as a Holy Book is very personal to many people. The author hopes that those who will endeavor to study the Qur'\=an using the methods and techniques presented in this paper will do so with the utmost respect and care. 