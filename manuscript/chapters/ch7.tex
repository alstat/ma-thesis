\chapter{SUMMARY AND CONCLUSIONS}\label{ch:conclusion}

This paper has demonstrated the use of Statistics and Machine Learning for analyzing the Qur'\=an. As emphasized in the introduction, the Qur'\=an is a text that has been studied for centuries, and it is essential to approach it with the utmost respect and care. The methodology presented in this work is not intended to replace traditional methods of studying the Qur'\=an, but rather to complement them by offering a new perspective on the text. The results of the analysis reveal that there are patterns and structures in the Qur'\=an that can be uncovered through statistical and machine learning techniques. These findings contribute to a deeper understanding of the text and its message.

The descriptive statistics of the counts of characters, words, and \arb[trans]{'AyAt} \arb{'AyAt}, when analyzed alongside the \arb[trans]{'asbAb 'alnuzUl} \arb{'asbAb 'alnuzUl}, show that the \arb[trans]{makkiyyaT} \arb{makkiyyaT} and \arb[trans]{madaniyyaT} \arb{madaniyyaT} \arb[trans]{suwar} \arb{suwar} form distinct clusters, which supports the H1 hypothesis postulated in Section~\ref{sec:hyp_structure}. The \arb[trans]{madaniyyaT} \arb{madaniyyaT} group exhibits a higher median and greater variability compared to the \arb[trans]{makkiyyaT} \arb{makkiyyaT} group in terms of word and character count, as well as within-\arb[trans]{'ayaT} \arb{'ayaT} variability.

Furthermore, the morphological analysis of the root word \arb[trans]{Alh} \arb[novoc]{Alh} in the Qur'\=an shows that its common form, \arb[trans]{'l-lah} \arb{'l-lah}, predominantly occurs in the \arb[trans]{madaniyyaT} \arb{madaniyyaT}, while the rarer forms appear exclusively in the \arb[trans]{makkiyyaT} \arb{makkiyyaT}. An investigation into extant manuscripts dating from 660–710~CE up to 1000~CE reveals that these morphological forms already had the correct \arb[trans]{rasm} \arb{rasm} (consonantal skeleton) from the earliest manuscripts, and have remained unaltered to the present day. This suggests that there is no evidence of editorial alteration that might imply a shift in oral transmission or a deliberate propagation or suppression of particular teachings. The investigations, which included the plausible insertions proposed by \citeA{sinai2020oqs}, found no manuscript evidence to support such claims. Therefore, it is reasonable to conclude that the orally transmitted Qur'\=an has been well preserved, with no indication of insertions or deletions. Had any such insertions occurred, they would likely be present across multiple manuscripts, but the absence of such evidence indicates high textual fidelity. These findings support the H2 hypothesis in Section~\ref{sec:hyp_structure} and the H6 and H7 hypotheses in Section~\ref{sec:hyp_manuscript_preservation}, which suggest that any differences are limited to orthographic conventions while maintaining a consistent consonantal skeleton.

Moreover, in the rhythmic analysis of the Qur'\=an, this paper proposed techniques for visualizing rhythmic patterns, including line plots, heatmaps, rhythmic graphs, and histograms with density plots. The results reveal consistent rhythmic patterns, with transition probabilities aligned with traditional understandings\footnote{That there is prolongation at the end of every \arb[trans]{'ayaT} \arb{'ayaT}, or that it rhymes from the beginning to the end of the Qur'\=an.} of the text. Specifically, about 70\% of the time, there is a transition from a short vowel to a long vowel in the final two syllables of an \arb[trans]{'ayaT} \arb{'ayaT}. These findings support the H3 hypothesis in Section~\ref{sec:hyp_structure}.

Additionally, the paper studied concentric structures in the Qur'\=an, focusing on \arb[trans]{sUraTu 'l-baqara} \arb{sUraTu 'l-baqara}, and confirmed the presence of concentrism. Rather than subjectively determining the structural boundaries of the concentric segments, the paper proposed a method to objectively identify these boundaries using a Genetic Algorithm. The algorithm was tested with two types of initial populations of 1{,}000 structural borders: one simulated from a Dirichlet distribution and the other from a discrete uniform distribution. Results show that, with proper padding, the discrete uniform population produces reasonable class boundaries, closely aligning with those identified by \citeA{farrin2014structure}. The algorithm used the CL-AraBERT model to extract Qur'\=anic word embeddings and employed cosine distance as the similarity metric.

Indeed, CL-AraBERT effectively captured semantic relationships, as the themes corresponding to the optimal structural borders identified by the Genetic Algorithm, $A\;(1\text{--}44)$, $B\;(45\text{--}86)$, $C\;(87\text{--}162)$, $D\;(163\text{--}163)$, $C^*\;(164\text{--}195)$, $B^*\;(196\text{--}236)$, $A^*\;(237\text{--}286)$, extracted using OpenAI's GPT-4o model, were found to form a concentric structure as validated by the author. This supports the H4 and H5 hypotheses in Section~\ref{sec:hyp_comp_linguist}.

Furthermore, this paper has shown that the Qur'\=an is a complex and multi-layered text that can be studied through various analytical methods. The findings provide new insights into the structure and meaning of the text and open new directions for research. The paper underscores the value of integrating traditional and modern methods in Qur'\=anic studies, demonstrating that these approaches can complement one another to offer deeper understanding. In particular, beyond the use of statistical and machine learning methods, the application of programming languages such as Julia—alongside libraries like \texttt{QuranTree.jl} \cite{asaad2021qurantree} and \texttt{Yunir.jl} \cite{al_ahmadgaid_b_asaad_yunir}—has proven to be highly effective for Qur'\=anic analysis. These findings support the H8 and H10 hypotheses in Section~\ref{sec:hyp_interdisciplinary}.

Finally, the paper also discussed the results from an Islamic philosophical perspective, emphasizing that all of these analytical efforts are ultimately meant to better understand the author behind the Qur'\=anic text. Errors in statistical models or tools should be interpreted as limitations in modeling rather than flaws in the text, and findings should be treated as hypotheses rather than definitive conclusions. This supports the H9 hypothesis in Section~\ref{sec:hyp_interdisciplinary}. Ultimately, philosophy remains the formal discipline for determining truth, and the logical and evidence-based approaches illustrated in this work serve to support Islamic philosophical inquiry in addressing profound questions through rational and empirical means.

It is the author's hope that this work inspires further research in this field, adopting computational techniques to study not only the Qur'\=an but also \arb[trans]{'a.hAdI_t} \arb{'a.hAdI_t} and other Islamic texts available through the \texttt{Kitab.jl} library \cite{al_ahmadgaid_b_asaad_kitab}. Additionally, the digitization efforts of the Corpus Coranicum project, which provides access to numerous early Qur'\=anic manuscripts, have been invaluable in advancing this line of research.

Lastly, the author wishes to reiterate that, like all sacred texts, the Qur'\=an holds deep personal significance for many. It is the author's sincere hope that those who engage with the Qur'\=an using the methods presented herein will do so with the utmost respect and care.
