\chapter{Summary and Conclusions}\label{ch:conclusion}
The paper has demonstrated the use of Statistics and Machine Learning for analyzing the Qur'\=an. As emphasized in the introduction of this paper, the Qur'\=an is a book that has been studied for centuries, and it is important to approach it with the utmost respect and care. The methodology used in this paper is not intended to replace traditional methods of studying the Qur'\=an, but rather to complement them by providing a new perspective on the text. The results of the analysis have shown that there are patterns and structures in the Qur'\=an that can be revealed through statistical and machine learning techniques. These findings can help deepen our understanding of the text and its message.

The descriptive statistics of the count of characters, words, and \arb[trans]{'AyAt} \arb{'AyAt} following the \arb[trans]{'Asbab al-nuzUl} \arb{'Asbab al-nuzUl}, has shown that the \arb[trans]{makkiyyaT} \arb{makkiyyaT} and \arb[trans]{madaniyyaT} \arb{madaniyyaT}\newline\arb[trans]{suwar} \arb{suwar} forms separate cluster, with \arb[trans]{madaniyyaT} \arb{madaniyyaT} having higher median as opposed to \arb[trans]{makkiyyaT} \arb{makkiyyaT}. This distinction does support the claims of traditional Muslims with regards to the \arb[trans]{'Asbab al-nuzUl} \arb{'Asbab al-nuzUl}, at the end of the day.