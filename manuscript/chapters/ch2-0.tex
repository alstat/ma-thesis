\chapter{Review of Related Literature}\label{ch:rrl}

As mentioned in the previous chapter, the earliest paper on Qur'\=anic studies using computer was likely the work of Rashad Khalifa in 1968\footnote{\url{https://www.masjidtucson.org/quran/miracle/a_profound_miracle_sura68nun133.html}}, which led to one of his book entitled `The Computer Speaks: God's Message to the World' (\textit{see} \citeA{rashad1981}). While the work of Rashad started at studying the mystery letters in the beginning of some \textit{s\=urahs} \arb{sUr} (for example Qur'\=an 2:1, 3:1, 7:1, etc.), it quickly went on to cover what he calls other \textit{mathematical miracles}, all of which are covered in \citeA{rashad1981}. His work were mainly based on counting the positional statistics of selected entities like names of God relative to other names or words in the Qur'\=an, and showing through counting and simple math of factoring some totals, he showed an interesting matches among these statistics. He was able to do this by transliterating the Qur'\=an into Alpha-Numeric typesets for easy parsing of the computers back then. His findings led him to generalize the claim that God revealed His words through this "mathematical" patterns throughout the Qur'\=an, and that those verses that were off and did not conform to this discovered "mathematical" patterns led him to extensive investigation of the said verses, and concluded that those could be or surely be an insertion that should not have been in the Qur'\=an in the first place. There are two verses that were off according to Rashad, and he called these verses as \textit{false verses}\footnote{\url{https://submission.org/App24.html}}, these are the last two \arb[trans]{'AyAt} \arb{'AyAt} of \arb[trans]{sUraTu 'l-tAwbaT} \arb{sUraTu 'l-tAwbaT} or The Chapter of \textit{Repentance}. These two verses were removed in Rashad's Qur'\=an translation\footnote{\url{https://www.masjidtucson.org/quran/frames/}}. Rashad believed so much on his findings that he self-proclaimed himself to be a messenger\footnote{\url{https://www.masjidtucson.org/submission/faq/rashad_khalifa_summary.html}} with this new findings and that the Qur'\=an nowadays should conform to his found "mathematical" patterns. This self-proclamation led to his assassination.

What the above tells us is that early on with the advent of computers, there was already interest in trying to understand the structural characteristics of the Qur'\=an. This is because the Qur'\=an is not like the Bible that is arranged chronologically, but rather arranged in generally decreasing number of \arb[trans]{'AyAt} \arb{'AyAt} per \arb[trans]{sUraT} \arb{sUraT} as shown in Figure \ref{fig:ayah_word_count}. This unique structure led to different literatures on understanding the Qur'\=an's design in terms of its unique presentation of topics, not only from the Islamic studies researchers but also computer scientists who were into Arabic Natural Language Processing studies. 

Among the pioneers in computational applications for the Qur'\=an is the work of \shortciteA{thabet2004}, who built a stemmer system for the Qur'\=an. A stemmer system is a system for trimming inflected words into its basic form, which grammatically mean its the root form. For example, in English language the root word for \textit{computational}, \textit{computer}, \textit{computation}, and \textit{computerize} is \textit{compute}. Therefore, the root of a word forms different \textit{stems} representing the different words. Hence, the idea of stemming is to trim these words into its basic form, so that it would be easy to do word clustering or grouping through word similarity. According to \citeA{thabet2004}, the rich morphology of the Qur'\=anic language or the Classical Arabic makes it even more difficult to do word stemming. Moving on, \citeA{thabet2005} builds on top of this stemming system, and used it for tokenization of the Qur'\=anic words, and building a statistical methodology for clustering or grouping the chapters of the Qur'\=an, in particular \citeA{thabet2005} used a Agglomerative Hierarchical Clustering based on the Euclidean distance of the adjusted word frequency of a \arb[trans]{sUraT} \arb{sUraT}.

The work of \citeA{thabet2004} and \citeA{thabet2005} used a Qur'\=an's corpus that was transliterated to Roman letters and symbols. Indeed, with the growing interests on studying the Qur'\=an from the lense of Data Analysis and Natural Language Processing, resulted into creating digital corpi of the Qur'\=an that captures the different aspects of its linguistic styles. Hence, a series of work by Sharaf and Atwell led to the following publications: \shortciteA{sharaf2009} studied knowledge representation of the Qur'\=an's verb valences using FrameNet frames, the output of which is a lexical database as a corpus of Qur'\=an's verbs. Further, the work of \shortciteA{sharaf2012} came up with corpus for the annotations of the Qur'\=anic pronouns, the authors named it as QurAna. Building on this work, \shortciteA{sharaf2012b} came up with a corpus for studying Qur'\=anic relatedness based on the commentary of Ibn Kathir \arb{ibn ka_tir}, the authors named this corpus as QurSim. Apart from Sharaf and Atwell, Dukes and Habash was also working on this, but specifically on the morphological annotations. The final and verified\footnote{The authors used online community collaboration for checking or verification of the morphological annotations. This was necessary since the Qur'\=an is a highly regarded book by Muslims, and there should be no mistakes in annotations.} work of the authors is the \shortciteA{dukes-habash-2010-morphological}, which also led to other publications \shortcite{dukes2010online, dukes2013supervised, dukes2010dependency} related to this.

With the establishment of a morphological annotated corpus for the Qur'\=an, the hoped was to have further analyses on the said scripture using statistical and machine learning methodologies. As such, the work of \shortciteA{dukes-habash-2011-one, dukes2015statistical} were among the first to do so, where they constructed a statistical parser through machine learning. This was then followed by \shortcite{siddiqui2013} who used the said corpus for topic modeling using Latent Dirichlet Allocation, the said study started with 114 \arb[trans]{sUwar} \arb{sUwar}, but after processing it went down to 24 \arb[trans]{sUwar} \arb{sUwar} after considering \arb[trans]{sUwar} \arb{sUwar} with 1000 or more words. This was mainly due to the very sparse document term matrix for the Term Frequency - Inverse Document Frequency\footnote{\textit{See} Section \ref{sec:tf-idf}} (TF-IDF) embedding if considering all of the 114 \arb[trans]{sUwar} \arb{sUwar}. Aside from this, the rest have used the corpus by \shortciteA{dukes-habash-2010-morphological} as part of benchmark for morphological analysis or for new lexicographic database, for example \shortciteA{sabtan2017morphological, jarrar-hammouda-2024-qabas-open}.

For the case of \textit{rhythmic analysis} of the Qur'\=an, most of the literatures have looked into the melodious recitation of this rhythmic feature of the Qur'\=an, for example the work of \shortciteA{samhani2022rhythms, shekha2013effects, abdullah2011effect} and \shortciteA{naqvi2020effect} have all investigated the effect of listening to the Qur'\=an recitation on temporal Electroencephalogram (EEG) signals and on human physiology through electrocardiogram (ECG) signal. The work of \shortciteA{samhani2022rhythms} compared the results of listening to Arabic News, Rest, and listening to the Qur\=an recitation \arb[trans]{sUraTu 'l-fAti.haT} \arb{sUraTu 'l-fAti.haT}. They used Analysis of Variance (ANOVA) and Hierarchical Clustering for the analyses on the Beta oscillation data. Their study found that the said Beta rhythms synchronization
with the \arb[trans]{sUraTu 'l-fAti.haT} \arb{sUraTu 'l-fAti.haT} is associated with verbal fluency, academic performance, social
interaction, inhibitory function, movement planning, self-motivation, self-management and
reactivation of sensory features of memory trace as highly activated cluster, followed by working memory, language processing and decision making as medially activated cluster. The same study was done by \shortciteA{shekha2013effects}, but instead of listening to Arab News and Qur'\=an, this time they used soft music, hard music, and listening to the Qur'\=an. The authors used t-test, and ANOVA to compare the results from the three input audio. Their findings suggest that the Qur'\=an recitation both during opened eyes or closed eyes while listening to the Qur'\=an have significant magnitude in \textit{alpha} waves electrode. The same findings was also found in the study of \shortciteA{abdullah2011effect}. Further, the work of \shortciteA{naqvi2020effect} investigated the effect of the Qur'\=an on ECG. The authors used different classification models for classifying signals from either with Qur'\=an listening and without.


are done through audio analysis. investigation, and this often lead to investigating portion of the Qur'\=an only. Examples of this are the work of \shortciteA{abalkheel2022linguistic}, who did rhythmic analysis of pauses or \textit{fawasil} in the Qur'\=an. Another is the work of \shortciteA{mahmood2023rhythmic}. Others have used the Qur'\=an audio recitation for the rhythm analyses. For example, .

Moving on, as for the theory of \textit{concentrism}, only \shortciteA{farrin2014structure} is the one who studied it, and none has ever looked into this theory using mathematical formulation and statistical analyses of it. 

Finally, as for the application of Retrieval-Augmented Generation (RAG), the following are the papers that have looked into it.