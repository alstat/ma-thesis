\chapter*{Abstract}
The interest of the paper is to provide a comprehensive text analytics of the Qur'an. This is by utilizing Statistical and Machine Learning methods to computationally analyze the said scripture. Specifically, the following procedures have been done: descriptive analyses of the structure of the Qur'an, morphological analyses of the Qur'an. Further, the data used for the Qur'anic Arabic corpus is the one in QuranTree.jl by Asaad (2022). The computational software used is Julia programming language.