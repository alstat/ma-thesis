\chapter{Literature Review}

As mentioned in the previous chapter, the earliest work in Qur'\=anic studies using computer was likely the work of Rashad Khalifa in 1968\footnote{\url{https://www.masjidtucson.org/quran/miracle/a_profound_miracle_sura68nun133.html}}, which led to one of his book entitled `The Computer Speaks: God's Message to the World' (\textit{see} \citeA{rashad1981}). While the work of Rashad started at studying the mystery letters in the beginning of some \textit{s\=urahs} \arb{sUr} (for example Qur'\=an 2:1, 3:1, 7:1, etc.), it quickly went on to cover what he calls other \textit{mathematical miracles}, all of which are covered in \citeA{rashad1981}. Rashad went on to claim that these findings meant that God revealed His words through this mathematical patterns through out the Qur'\=an, and that those verses that were off and did not conform to this discovered mathematical patterns led him to extensive investigation of the said verses, and concluded that those could be or surely be an insertion that should not have been in the Qur'\=an in the first place. There are two verses that were off, and Rashad called these verses as \textit{false verses}\footnote{\url{https://submission.org/App24.html}}, these are the last two \arb[trans]{'ayAt} \arb{'ayAt} of \arb[trans]{sUraT 'l-tAwbaT} \arb{sUraT 'l-tAwbaT} or The Chapter of \textit{Repentance}. These two verses were removed in Rashad's Qur'\=an translation\footnote{\url{https://www.masjidtucson.org/quran/frames/}}. Rashad believed so much on his findings that he claimed to be a messenger\footnote{\url{https://www.masjidtucson.org/submission/faq/rashad_khalifa_summary.html}} with this new findings and that the Qur'\=an nowadays should conform to his found mathematical patterns. This self-proclamation led to his assassination.

Fast forward to 20th century, \citeA{thabet2004} started developing a stemmer system for the Qur'\=an. A stemmer system is a system for trimming inflected words into its basic form, which is the root. For example, in English the root word for \textit{computational}, \textit{computer}, \textit{computation}, and \textit{computerize} is \textit{compute}. Therefore, from the root forms different stems representing the different words. Hence, the idea of stemming is to trim these words into its basic form. The use case of stemming is finding groups of words, which by using the root of the word makes it easy to find relations or similarity. This is the work done by \citeA{thabet2004}, which according to the author it is even more a challenge for Arabic words since it is highly inflected, and more so for the Classical Arabic texts like the Qur'\=an. 

Building on the work of \citeA{thabet2004}, \citeA{thabet2005} used a statistical methodology for clustering the chapters of the Qur'\=an, in particular using the agglomerative hierarchical clustering. The data processing makes use of the stemming methodology in \citeA{thabet2004} to remove the different inflections on the Qur'\=anic words. Moving on, the work of \citeA{noordin2006} focused on information retrieval of Qur'\=anic content by surveying 125 websites and investigating how Qur'\=anic informations are presented, their aim is to propose a system for retrieving these information.

The work of \citeA{sharaf2009} studied knowledge representation of the Qur'\=anic verb valences using FrameNet frames, the output of which is a lexical database of the corpus of Qur'\=an verbs. Further, the work of \citeA{sharaf2012} came up with corpus for the annotations of the Qur'\=anic pronouns, the authors named it as QurAna. Building on this work, \citeA{sharaf2012b} came up with a corpus for studying Qur'\=anic relatedness based on the commentary of Ibn Kathir \arb{ibn ka_tir}, the authors named this corpus as QurSim.

Moving on, an unpublished work by \citeA{nassourou2011} used Machine Learning to study.