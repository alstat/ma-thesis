\chapter{Literature Review}

As mentioned in the previous chapter, the earliest work in Qur'\=anic studies using computer was likely the work of Rashad Khalifa in 1968\footnote{\url{https://www.masjidtucson.org/quran/miracle/a_profound_miracle_sura68nun133.html}}, which led to one of his book entitled `The Computer Speaks: God's Message to the World' (\textit{see} \citeA{rashad1981}). While the work of Rashad started at studying the mystery letters in the beginning of some \textit{s\=urahs} \arb{sUr} (for example Qur'\=an 2:1, 3:1, 7:1, etc.), it quickly went on to cover what he calls other \textit{mathematical miracles}, all of which are covered in \citeA{rashad1981}. Rashad went on to claim that these findings meant that God revealed His words through this mathematical patterns through out the Qur'\=an, and that those verses that were off and did not conform to this discovered mathematical patterns led him to extensive investigation of the said verses, and concluded that those could be or surely be an insertion that should not have been in the Qur'\=an in the first place. There are two verses that were off, and Rashad called these verses as \textit{false verses}\footnote{\url{https://submission.org/App24.html}}, these are the last two \arb[trans]{'ayAt} \arb{'ayAt} of \arb[trans]{sUraT 'l-tAwbaT} \arb{sUraT 'l-tAwbaT} or The Chapter of \textit{Repentance}. These two verses were removed in Rashad's Qur'\=an translation\footnote{\url{https://www.masjidtucson.org/quran/frames/}}. Rashad believed so much on his findings that he claimed to be a messenger\footnote{\url{https://www.masjidtucson.org/submission/faq/rashad_khalifa_summary.html}} with this new findings and that the Qur'\=an nowadays should conform to his found mathematical patterns. This self-proclamation led to his assassination.

\begin{table}[!h]
    \begin{flushleft}
        \caption{Summary of papers on computational Qur'\=anic studies}
        \begin{tabularx}{\textwidth}{YYYY}
            \toprule
            \textbf{Research} & \textbf{Computing Field} &\textbf{Computing Methdology} & \textbf{Output}\\\midrule
            \citeA{thabet2004}&Linguistic Analysis &Stemming of transliterated Qur'\=anic words&Qur'\=an stemmer\\
            \citeA{thabet2005}&Data Mining&Statistical heirarchical cluster analysis of Qur'\=an's chapters&Clusters of Qur'\=an chapters\\
            \citeA{noordin2006}&Information Retrieval&Retrieving Qur'\=anic content and important knowledge derived from the Qur'\=an& Qur'\=anic retrieval systemm\\
            \bottomrule
        \end{tabularx}
    \end{flushleft}
    \label{tab:single_ave}
\end{table}


The Quranic Arabic Corpus (Dukes and Habash, 2010) provides a complete annotation of the Morphological Features of the Qur'\=an. In an effort to make the corpus accessible computationally, Dukes and Habash (2010) developed the Java package, JQuranTree. Therefore, the next steps from Dukes and Habash (2010) is to make use of this data and start applying the computational methodologies to study further the Qur'\=an. \cite{sinai2017} also started looking into the numerical patterns but only looking at the numerical count of the ayah and surah. Although, his approach does not make use of computational approaches. Indeed, \cite{sinai2017} count statistics is similar in approach to what Rashad Khalifa (1967) done with the mysterious letters in the Qur'\=an, like Qur'\=an 2:1. 

