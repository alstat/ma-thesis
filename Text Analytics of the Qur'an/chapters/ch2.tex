\chapter{Literature Review}

As mentioned in the previous chapter, the earliest work in Qur'\=anic studies using computer was likely the work of Rashad Khalifa in 1968\footnote{\url{https://www.masjidtucson.org/quran/miracle/a_profound_miracle_sura68nun133.html}}, which led to one of his book entitled `The Computer Speaks: God's Message to the World' (\textit{see} \citeA{rashad1981}). While the work of Rashad started at studying the mystery letters in the beginning of some \textit{s\=urahs} \arb{sUr}, it quickly went on to cover what he calls other \textit{mathematical miracles}, all of which are covered in \citeA{rashad1981}.

\begin{table}[!h]
    \centering
    \caption{Number of observations in each cluster with respect to the linkage}
    \begin{tabularx}{\textwidth}{c|YYYYY}
        \toprule
        \multirow{2}{*}{\textbf{Linkage}} & \multicolumn{5}{c}{\textbf{Clusters}}\\\cline{2-6}
        &A & B & C & D & E \\\midrule
        Single & 1831 & 1 & 1 & 1 & 1\\
        Centroid & 549 & 528 & 7 & 746 & 5\\
        \bottomrule
    \end{tabularx}
    \label{tab:single_ave}
\end{table}


The Quranic Arabic Corpus (Dukes and Habash, 2010) provides a complete annotation of the Morphological Features of the Qur'\=an. In an effort to make the corpus accessible computationally, Dukes and Habash (2010) developed the Java package, JQuranTree. Therefore, the next steps from Dukes and Habash (2010) is to make use of this data and start applying the computational methodologies to study further the Qur'\=an. \cite{sinai2017} also started looking into the numerical patterns but only looking at the numerical count of the ayah and surah. Although, his approach does not make use of computational approaches. Indeed, \cite{sinai2017} count statistics is similar in approach to what Rashad Khalifa (1967) done with the mysterious letters in the Qur'\=an, like Qur'\=an 2:1. 

